\documentclass[a4paper,landscape]{foils}
\usepackage{graphicx}
\usepackage{amsfonts}
\usepackage{amsmath}

\setlength{\foilheadskip}{0cm}
\setlength{\textwidth}{25cm}
\setlength{\textheight}{17cm}
\setlength{\oddsidemargin}{-1.0cm}
\setlength{\evensidemargin}{-1.0cm}
\setlength{\headsep}{0cm}
\setlength{\topmargin}{-1cm}

\title{Explaining Uninstallable Debian Packages}

\author{
  \vspace{1cm} \\
  \centerline{
    \includegraphics[height=3cm]{uqlogo.jpg} \hfil
    \includegraphics[height=3cm]{kvoa.png}
  } \\ 
  \vspace{2cm} \\
  \centerline{
    \parbox{10cm}{Richard Cole \\ \texttt{<rcole@itee.uq.edu.au>}} \hfil
    \parbox{10cm}{Robert J\"{a}schke \\ }
  }
}
\date{}

\MyLogo{} 
\newcommand\slide[1]{\foilhead[-1cm]{#1}}

\begin{document}

\maketitle

\slide{Motivating example}

{\small \begin{verbatim}
pluto:/# apt-get install cdebconf
Reading package lists... Done
Building dependency tree... Done
Some packages could not be installed. This may mean that you have
requested an impossible situation or if you are using the unstable
distribution that some required packages have not yet been created
or been moved out of Incoming.

Since you only requested a single operation it is extremely likely that
the package is simply not installable and a bug report against
that package should be filed.
The following information may help to resolve the situation:

The following packages have unmet dependencies:
  cdebconf: Depends: libgtk2.0-0 (>= 2.6.0) but it is not going to be installed
            Depends: libpango1.0-0 (>= 1.8.2) but it is not going to be installed
E: Broken packages
\end{verbatim}
}

\includegraphics[width=0.8 \textheight, angle=270]{c2hs-example.pdf}


\slide{Interesting objectives}

\begin{enumerate}
\item helping maintainers, developers, users to resolve conflicts in packaging system
\item automatically find uninstallable packages
\item \emph{explain}, why a package is uninstallable - both graphically and in words
\item give some minimal set of ``guilty'' packages
\item guidance in resolving conflict (``what happens, if I remove that dependency?'')
\end{enumerate}



\slide{Debian packaging system}

\begin{enumerate}
\item divided into three distributions: \emph{stable}, \emph{testing}, \emph{unstable}
\item \emph{stable}: no new packages, just security updates (or packages with grave bugs)
\item \emph{testing} becomes \emph{stable} after a freeze (last time: June 6th, 3.1 \emph{Sarge}, 2002, 3.0\emph{Woody})
\item few restrictions for a package to get into \emph{unstable}
\item transition \emph{unstable} -> \emph{testing}: 
\begin{itemize}
\item in sync on all distributions
\item no dependencies, that makes package uninstallable
\item fewer release-critical bugs than the version currently in \emph{testing}
\item available in \emph{unstable} for at 2, 5, 10 days (depending in urgency)
\item must not break dependency of any package in \emph{testing}
\item depending packages already in \emph{testing} or accepted at the same time
\end{itemize}
\item broken, i.e. \emph{uninstallable} packages only in \emph{unstable}
\item bug reporting system: everbody can participate, has influence on package transition (``\emph{release critical bugs}'')
\item 16700 packages in unstable, several hundred maintainers
\item most interesting thing: it works!
\item social aspect: community of several thousend people enhancing Debian all the time, working together
trough mailing lists, bug tracking system, CVS and packaging system
\end{enumerate}

\slide{Package Metadata}

{\small \begin{verbatim}
Package: cdebconf
Priority: extra
Section: utils
Maintainer: Debian Install System Team <debian-boot@lists.debian.org>
Architecture: i386
Version: 0.84
Provides: debconf-2.0
Depends: libatk1.0-0 (>= 1.9.0), libc6 (>= 2.3.5-1), 
  libdebian-installer4 (>= 0.31), libglib2.0-0 (>= 2.6.0), 
  libgtk2.0-0 (>= 2.6.0), libnewt0.51, libpango1.0-0 (>= 1.8.2), libtextwrap1
Conflicts: debconf, debconf-doc
Filename: pool/main/c/cdebconf/cdebconf_0.84_i386.deb
Size: 129306
MD5sum: 40ad3744b2c7795c8fc8de647e657c64
Description: Debian Configuration Management System (C-implementation)
 Debconf is a configuration management system for Debian packages. It is ...
 .
 Installing this package is rather dangerous now. It will break debconf.
 You have been warned!
Tag: devel::packaging, suite::debian, uitoolkit::gtk
\end{verbatim}}

\slide{Package Metadata - Continued}

\begin{enumerate}
\item Description, Maintainer, Tag
\item Package, Section, Installed-Size, Size, MD5Sum
\item Priority, Architecture, Version, Provides, Depends, Conflicts, Filename
\item packages which \emph{have to be} installed and which \emph{can't} be installed together with particular package
\item \emph{conflicts} arise, when a package (transitively) depends on some non-existing package or when it depends on some package which is ``somehow'' in conflict
\item recently: \emph{tagging} of packages (we wrote some small FCA application)
\end{enumerate}

\slide{Architecture}

on the white board

\slide{Conversion to CNF}

\noindent \fbox{\begin{minipage}{12cm} {\tiny \tt \begin{tabbing}
\hspace{1em}\= Package: kdelibs4-dev
\\\> Priority: optional
\\\> Section: libdevel
\\\> Architecture: i386
\\\> Version: 4:3.4.1-1
\\\> Depends: kdelibs4 (= 4:3.4.1-1),
\\\>\hspace{0.5em}\= kdelibs-bin (= 4:3.4.1-1),
\\\>\> libart-2.0-dev, libarts1-dev (>= 1.4.0), 
\\\>\> libasound2-dev, libaspell-dev,
\\\>\> libbz2-dev, libcupsys2-dev, 
\\\>\> libfam-dev, libidn11-dev, ...
\\\> Conflicts: kdelibs-dev (<< 4:3.0.0),
\\\>\> kdelibs3 (<< 4:3.0.0), kdelibs3-bin (<< 4:3.0.0), 
\\\>\> kdepim-libs (<< 4:3.0.0), libarts (<< 4:3.0.0), ...
\end{tabbing}}\end{minipage}}
%
\begin{minipage}{12cm} 
\begin{itemize}
\item Conjuctive Normal Form: set of disjunctions
\item Each element is a disjunction is a literal 
      (either a ground term or negated ground term)
\item Packages with sepecific versions become grounded terms
\end{itemize}
\end{minipage}

\noindent * Dependencies are as conjunction of version expressions \\
 * Each expression can match multiple package versions \\
 * Thus a conjunction of disjunctions \\
 * Conflicts are always between two packages \\

\slide{SAT Solvers}

\begin{itemize}
\item Find valuation function from variable $V$ to $\{true,false\}$ that make the clauses true.
\item Return UNSAT if there is no valuation function that makes all of the functions true.
\item How do they work? (Array of clauses, current state, search for a valuation function with backtracking)
\item What makes them faster? (Order in which variables are choosen, inverted lists for clauses containing variables).
\item Complexity: NP-Complete (whatever that means, worst case complexity is exponential in the number of variables).
\item Complexity: Linear in the case of horn clauses (at most one positive literal). We don't have horn clauses.
\end{itemize}

\slide{Min-UNSAT}

\begin{enumerate}
\item \emph{minimally-unsatisfiable subformulas} (MUS) of an unsatisfiable CNF formula $\varphi$
\item in general NP-hard
\item two implementations, both calculating \emph{maximally satisfiable subformulas} (MSS) first
\item breadth-first vs. depth-first (last one faster)
\item example: 
\begin{enumerate}
  \item $\varphi = (x_1) \land (\overline{x_1}) \land (\overline{x_1} \lor x_2) \land (\overline{x_2})$
  \item $\operatorname{coMSS}(\varphi) = \bigl\{
      \{(x_1)\}, 
      \{(\overline{x_1}),(\overline{x_1} \lor x_2)\},
      \{(\overline{x_1}),(\overline{x_2})\} \bigr\}$
  \item $\operatorname{MUS}(\varphi) = \bigl\{
      \{(x_1),(\overline{x_1})\},
      \{(x_1),(\overline{x_1} \lor x_2), (\overline{x_2})\}\bigr\}$
\end{enumerate}
\end{enumerate}

\slide{Graph representation of CNF for Visualisation}

\begin{itemize}
\item Convert $A \rightarrow B$ into a simple edge
\item Convert $A \rightarrow B \vee C$ into a hyperedge
\item Convert $A \rightarrow \neg B$ into dotted edge.
\end{itemize}

Cannot hope to visualise a large clause set. It is much too big.

\slide{Relevent Clauses: When installing a package which clauses are relevent?}

\begin{itemize}
\item If we want to install cdebconf-1.82 which clauses are relevent?
\item All those containing the variable cdebconf-1.82
\item Any dependencies of cdebconf-1.82, i.e. $A \rightarrow B \vee C$
\item Any conflicts, i.e. $A \rightarrow \neg B$
\item We need to recursively collect relevent variables and clauses
\item Don't need to recurse through conflicts to the variables, just collect the clauses.
\end{itemize}

This reduces the number of clauses from 233,000 to less than about
4000 for each package. Often the number of relevent clauses is very
small $< 10$.

\slide{Graph Reduction}

\noindent The clause graph can be simplified in two ways:

\begin{itemize} 
\item Verticies that have the same in and out edges can be collapsed
\item This happens more than you would think
\item Vertices that are in a chain, one input and one output can be collapsed.
\item This reduces the number of variables and clauses needed to the UNSAT computation.
\item Also makes the graph easier to visualise because it has fewer nodes and fewer edges.
\end{itemize}






\end{document}